\section{Components}

\subsection{Sorts}

A component's definitions define objects, each of which is of one of
the following \emph{sort}s:

\begin{sum-productions}
  \production{coresort}
    \CSFUNC | \CSTABLE | \CSMEMORY | \CSGLOBAL | \CSTYPE | \CSMODULE |
    \CSINSTANCE
  \production{sort}
    \SCORE~\coresort \alt
    \SFUNC | \SVALUE | \STYPE | \SCOMPONENT | \SINSTANCE
\end{sum-productions}

\subsection{Indices}

Each object defined by a component exists within an \emph{index space}
made up of all objects of the same sort. Unlike in Core WebAssembly, a
component definition may only refer to objects that were defined prior
to it in the current component. Future definitions refer to past
definitions by means of an \emph{index} into the appropriate index
space:

\begin{record-productions}
  \production{coremoduleidx} \u32
  \production{coreinstanceidx} \u32
  \production{componentidx} \u32
  \production{instanceidx} \u32
  \production{funcidx} \u32
  \production{corefuncidx} \u32
  \production{valueidx} \u32
  \production{typeidx} \u32
  \production{coretypeidx} \u32
\end{record-productions}

\begin{record-productions}
  \production{coresortidx} \{ \CSISORT~\coresort, \CSIIDX~\u32 \}\\
  \production{sortidx} \{ \SISORT~\sort, \SIIDX~\u32 \}
\end{record-productions}

\subsection{Definitions}

\begin{sum-productions}
  \production{definition}
    \DCOREMODULE~\core:module \alt
    \DCOREINSTANCE~\coreinstance \alt
    \DCORETYPE~\coredeftype \alt
    \DCOMPONENT~\component \alt
    \DINSTANCE~\instance \alt
    \DALIAS~\alias \alt
    \DTYPE~\deftype \alt
    \DCANON~\canon \alt
    \DSTART~\start \alt
    \DIMPORT~\import \alt
    \DEXPORT~\export
\end{sum-productions}

\subsection{Core Instances}

A core instance may be defined either by instantiating a core module
with other core instances taking the place of its first-level imports,
or by creating a core instance from whole cloth by combining core
definitions already present in our index space:

\begin{sum-productions}
  \production{coreinstance}
    \CIINSTANTIATE~\coremoduleidx~\coreinstantiatearg^{*} \alt
    \CIEXPORTS~\coreexport^{*}
  \production{coreinstantiatearg}
    \{ \CIANAME~\name, \CIAINSTANCE~\coreinstanceidx \}
  \production{coreexport} \{ \CENAME~\name, \CEDEF~\coresortidx \}
\end{sum-productions}

\subsection{Components}

A component is merely a sequence of definitions:

\begin{record-production}{component}
  \definition^{*}
\end{record-production}

\subsection{Instances}

Component-level instance declarations are nearly identical to
core-level instance declarations, with the caveat that more sorts of
definitions may be supplied as imports:

\begin{sum-productions}
  \production{instance}
    \IINSTANTIATE~\componentidx~\instantiatearg^{*}\\&&|&
    \IEXPORTS~\export^{*}\\
  \production{instantiatearg}
    \{ \IANAME~\name, \IAARG~\sortidx \}
\end{sum-productions}

\subsection{Aliases}

An alias definition copies a definition from some other module,
component, or instance into an index space of the current component:

\begin{sum-productions}
  \production{alias} \{ \ASORT~\sort, \ATARGET~\aliastarget \}
  \production{aliastarget}
    \ATEXPORT~\instanceidx~\name \alt
    \ATCOREEXPORT~\coreinstanceidx~\name \alt
    \ATOUTER~\u32~\u32
\end{sum-productions}

\subsection{Canonical Definitions}

Canonical definitions are the only way to convert between Core
WebAssembly functions and component-level shared-nothing functions
which produce and consume values of type $\valtype$. A \emph{canon
  lift} definition converts a core WebAssembly function into a
component-level function which may be exported or used to satisfy the
imports of another component; a \emph{canon lower} definition converts
an lifted function (often imported) into a core function.

\begin{sum-productions}
  \production{canon}
    \CLIFT~\core:funcidx~\canonopt^{*}~\typeidx \alt
    \CLOWER~\funcidx~\canonopt^{*}
  \production{canonopt}
    \COSTRINGENCODINGUTFEIGHT \alt
    \COSTRINGENCODINGUTFSIXTEEN \alt
    \COSTRINGENCODINGLATINONEUTFSIXTEEN \alt
    \COMEMORY~\core:memidx \alt
    \COREALLOC~\core:funcidx \alt
    \COPOSTRETURN~\core:funcidx
\end{sum-productions}

\subsection{Start definitions}

A start definition specifies a component function which this component
would like to see called at instantiation type in order to do some
sort of initialization.

\begin{record-production}{start}
  \{ \FFUNC~\funcidx, \FARGS~\valueidx^{*} \}
\end{record-production}

\subsection{Imports}

Since an imported value is described entirely by its type, an actual
import definition is effectively the same thing as an import
declaration:

\begin{record-production}{import}
  \importdecl
\end{record-production}

\subsection{Exports}

An export definition is simply a name and a reference to another
definition to export:

\begin{record-production}{export}
  \{ \ENAME~\name, \EDEF~\sortidx \}
\end{record-production}
