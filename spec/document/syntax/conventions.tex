\section{Conventions}

The WebAssembly component specification defines a language for
specifying components, which, like the WebAssembly core language, may
be represented by multiple complete representations (e.g. the
\hyperref[sec:binary-format]{binary format} and the
\hyperref[sec:text-format]{text format}). In order to avoid
duplication, the static and dynamic semantics of the WebAssembly
component model are instead defined over an abstract syntax.

The following conventions are adopted in defining grammar rules for abstract syntax.

\begin{itemize}
\item Terminal symbols (atoms) are written in sans-serif font:
  $\K{i32}, \K{end}$.

\item Nonterminal symbols are written in italic font:
  $\X{valtype}, \X{instr}$.

\item $A^n$ is a sequence of $n\geq 0$ iterations of $A$.

\item $A^\ast$ is a possibly empty sequence of iterations of $A$.
  (This is a shorthand for $A^n$ used where $n$ is not relevant.)

\item $A^+$ is a non-empty sequence of iterations of $A$. (This is a
  shorthand for $A^n$ where $n \geq 1$.)

\item $A^?$ is an optional occurrence of $A$. (This is a shorthand for
  $A^n$ where $n \leq 1$.)

\item Productions are written $\X{sym} ::= A_1 ~|~ \dots ~|~ A_n$.

\item Large productions may be split into multiple definitions,
  indicated by ending the first one with explicit ellipses,
  $\X{sym} ::= A_1 ~|~ \dots$, and starting continuations with
  ellipses, $\X{sym} ::= \dots ~|~ A_2$.

\item Some productions are augmented with side conditions in
  parentheses, ``$(\iff \X{condition})$'', that provide a shorthand for
  a combinatorial expansion of the production into many separate
  cases.

\item If the same meta variable or non-terminal symbol appears
  multiple times in a production, then all those occurrences must
  have the same instantiation. (This is a shorthand for a side
  condition requiring multiple different variables to be equal.)
\end{itemize}
